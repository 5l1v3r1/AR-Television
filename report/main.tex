%%%%%%%%%%%%%%%%%%%%%%%%%%%%%%%%%%%%%%%%%%%%%%%%%%%%%%%%%%%%%%%%%%%%%%
% LaTeX Example: Project Report
%
% Source: http://www.howtotex.com
%
% Feel free to distribute this example, but please keep the referral
% to howtotex.com
% Date: March 2011 
% 
%%%%%%%%%%%%%%%%%%%%%%%%%%%%%%%%%%%%%%%%%%%%%%%%%%%%%%%%%%%%%%%%%%%%%%
% How to use writeLaTeX: 
%
% You edit the source code here on the left, and the preview on the
% right shows you the result within a few seconds.
%
% Bookmark this page and share the URL with your co-authors. They can
% edit at the same time!
%
% You can upload figures, bibliographies, custom classes and
% styles using the files menu.
%
% If you're new to LaTeX, the wikibook is a great place to start:
% http://en.wikibooks.org/wiki/LaTeX
%
%%%%%%%%%%%%%%%%%%%%%%%%%%%%%%%%%%%%%%%%%%%%%%%%%%%%%%%%%%%%%%%%%%%%%%
% Edit the title below to update the display in My Documents
%\title{Project Report}
%
%%% Preamble
\documentclass[paper=a4, fontsize=14pt]{scrartcl}
\usepackage[T1]{fontenc}
\usepackage{fourier}

\usepackage[english]{babel}                                                         % English language/hyphenation
\usepackage[protrusion=true,expansion=true]{microtype}  
\usepackage{amsmath,amsfonts,amsthm} % Math packages
\usepackage[pdftex]{graphicx}   
\usepackage{url}
\usepackage{setspace}
\usepackage{geometry}
\geometry{left=2.5cm,right=2.5cm,top=2.5cm,bottom=2.5cm}
\usepackage{indentfirst} 
\setlength{\parindent}{2em}
%%% Custom sectioning
\usepackage{sectsty}
\allsectionsfont{\centering \normalfont\scshape}


%%% Custom headers/footers (fancyhdr package)
\usepackage{fancyhdr}
\pagestyle{fancyplain}
\fancyhead{}                                            % No page header
\fancyfoot[L]{}                                         % Empty 
\fancyfoot[C]{}                                         % Empty
\fancyfoot[R]{\thepage}                                 % Pagenumbering
\renewcommand{\headrulewidth}{0pt}          % Remove header underlines
\renewcommand{\footrulewidth}{0pt}              % Remove footer underlines
\setlength{\headheight}{13.6pt}


%%% Equation and float numbering
\numberwithin{equation}{section}        % Equationnumbering: section.eq#
\numberwithin{figure}{section}          % Figurenumbering: section.fig#
\numberwithin{table}{section}               % Tablenumbering: section.tab#

\setstretch{1.6}

%%% Maketitle metadata
\newcommand{\horrule}[1]{\rule{\linewidth}{#1}}     % Horizontal rule

\title{
        %\vspace{-1in}  
        \usefont{OT1}{bch}{b}{n}
        \normalfont \normalsize \textsc{16811 Math Fundamentals for Robotics\\
        Final Project Report} \\ [25pt]
        \horrule{0.5pt} \\[0.4cm]
        \huge Real-time Virtual Television in Augmented Reality \\
        \horrule{2pt} \\[0.5cm]
}
\author{
        \normalfont
        %\normalsize
        Kai Yu (kaiy1)\\
        Zhongxu Wang (zhongxuw)\\
        Ruoyuan Zhao (ruoyuanz)\\
        Qiqi Xiao (qiqix)\\[-3pt]       
        \\
        \normalsize
        \today
}
\date{}


%%% Begin document
\begin{document}

\maketitle
%xqq
\section{Abstract}

Augmented Reality(AR) is a group of applications that uses computer vision to analyze the real world and overlays virtual objects onto it. 
When wearing the AR device(i.e. an AR glass like Hololens), you don't need a real television or a screen connected to your PC. 
You can simply have a virtual television on any surface in the world, with its size adjustable, content specifiable, and able to be paused automatically when you want to stop for a while.
Besides, you don't need to buy any super high-resolution television any more: just project a virtual television on a really large wall and watch it!
In this project, our group implemented an augmented reality application from scratch. 
It projects a real-time virtual television with live video content onto a user-specified surface. 

%xqq
\newpage
\section{Introduction}\label{intro}
In this project, we built the basic algorithm of this augmented reality application.
And we also use some pre-recorded video for demonstration in stead of a real AR device, which can be harder to obtain and operate. 
A video shoots the scene that contain a surface, and our program can produce a new video stream that has a virtual television projected onto the surface, with video pre-stored video content playing in it.
For a real AR device, this video can be replaced by the new video generated by the AR light engine.

Basically, there are two parts: perform 3D reconstruction inside the system based on 2D camera only, and then project a virtual television onto the scene. 

The 3D reconstruction is based on simultaneous localization and mapping, which is as known as SLAM. In this project, we adopt the PTAM-ORB-SLAM framework. 
PTAM represents a multi-threaded key frame based on SLAM.
And ORB features are used for key point detection and matching. Besides, simultaneous localization and mapping (SLAM) is the computational problem of constructing or updating a map of an unknown environment while simultaneously keeping track of an agent's location within it. 
In our project, we extract the key idea of SLAM to estimate the 3D key points.

The key step of 3D reconstruction is the triangulation.
Generally, the configuration contains two sensors.
The projection centers of the sensors and the point on the object's surface can define a triangle.
Within this triangle, the distance between the sensors must be known.
Determining the angles between the sensors, the intersection point and the 2D coordinate can be calculated.

Also, the 3D estimate process may not be accurate enough, thus we apply another key step: Bundle Adjustment.
It can optimize the rough solution given by triangulation.
However, bundle adjustment is solving a nonlinear least square problem, thus takes much longer time.
The PTAM framework is designed to mitigate this problem.
It creates a new thread for bundle adjustment, enabling the system to estimate extrinsic metrics in real-time.

In this project, we combine the above knowledge and build an AR system based on these knowledge.
In section \ref{background}, we introduced some basic background knowledge, including ORB, SLAM, Bundle Adjustment and Canny algorithm.
In section \ref{implementation}, some implementation details of this project is elaborated.
And in section \ref{discussion}, we summarize our project and propose some further possible improvement.


\section{Background Knowledge}\label{background}

%zry
\subsection{ORB} find points

%zry
\subsection{SLAM} estimate 3d points

%xqq
\subsection{BA}

%xqq
\subsection{Canny} find edges, cross product

\section{Implementation Details}\label{implementation}
%xqq
\subsection{Pipeline}
The pipeline contains several steps: record video and intrinsic matrix of camera; extract key points based on orb feature extractor; match key points using KNN method; pick up key frames according to distances; reconstruct 3d cloud points using triangulation; use bundle adjustment to optimize projection matrix and 3d points; project virtual television to the surface. Our implementation is based on Ptam-Orb-Slam Framework. This project is easily extendable as well as contains a lot of math knowledge.

%zry
\subsection{ORB} find points

%zry
\subsection{SLAM} estimate 3d points

%xqq
\subsection{BA}

%xqq
\subsection{Canny} find edges, cross product

%zry
\subsection{Projection to the plane}

%zry
\section{Discussion}\label{discussion}


\bibliographystyle{plain}
\bibliography{references}

%%% End document
\end{document}